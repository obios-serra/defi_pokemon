\documentclass{article}
\usepackage{graphicx}
\usepackage[a4paper,margin=1in]{geometry}
\usepackage{hyperref}
\usepackage{listings}
\usepackage{xcolor}

\title{Decentralized Pokémon Trading Platform}
\author{Michele Serra (17-265-323), Elric Pascali (21-917-984)}
\date{31st of March 2025}

\lstset{
  basicstyle=\ttfamily\small,
  breaklines=true,
  frame=single,
  backgroundcolor=\color{gray!10},
  keywordstyle=\color{blue},
  commentstyle=\color{green!50!black},
  stringstyle=\color{orange},
  showstringspaces=false
}

\begin{document}

\maketitle

\section{Overview}
This project is a decentralised application (dApp) that enables users to mint, own, and trade Pokémon cards as Non-Fungible Tokens (NFTs) on the Ethereum blockchain. It is built using Solidity (smart contracts), Hardhat (development environment), React (frontend), and Ethers.js (blockchain interaction).

\section{Project Architecture}
\subsection{Components}
\begin{itemize}
    \item \textbf{Smart Contracts (Solidity, Hardhat)}
    \begin{itemize}
        \item \texttt{PokemonNFT.sol}: Implements the ERC-721 standard for Pokémon NFTs.
        \item \texttt{PokemonMarket.sol}: Implements a marketplace for fixed-price listings and auctions.
        \item Key features:
        \begin{itemize}
            \item Minting, transferring, and pausing NFTs.
            \item Commit-reveal mechanism for secure minting.
            \item Fixed-price and auction-based trading.
        \end{itemize}
    \end{itemize}
    
    \item \textbf{Backend (Hardhat Local Testnet)}
    \begin{itemize}
        \item Provides a local Ethereum blockchain for development and testing.
    \end{itemize}
    
    \item \textbf{Frontend (React, Ethers.js, MetaMask Integration)}
    \begin{itemize}
        \item Connects to MetaMask for wallet authentication.
        \item Allows users to mint, transfer, and trade Pokémon NFTs.
        \item Displays owned Pokémon and active marketplace listings.
    \end{itemize}
\end{itemize}

\section{Setup Instructions}
\subsection{Prerequisites}
Ensure the following are installed:
\begin{itemize}
    \item Node.js (v18 or v20 recommended)
    \begin{lstlisting}
    brew install node
    \end{lstlisting}
    \item \textbf{MetaMask browser extension} (\href{https://metamask.io/}{MetaMask}): Since the dApp runs in a local test environment, you need to configure MetaMask to connect to a custom local network instead of the Ethereum Mainnet. Open MetaMask and click on the network selector at the top of the interface. In the dropdown, click on \texttt{+ Add a custom network}. In the dialog that opens, enter the following details:

    \begin{itemize}
      \item \textbf{Network name}: \texttt{Localhost 8545}
      \item \textbf{RPC URL}: \texttt{http://127.0.0.1:8545}
      \item \textbf{Chain ID}: \texttt{1337}
      \item \textbf{Currency symbol}: \texttt{ETH}
    \end{itemize}
    
    Click \texttt{Save} to add the network. Then, switch to this custom network by selecting it from the network dropdown. If the "Show test networks" option is not enabled, you can   activate it under \texttt{Settings > Advanced > Show test networks}.
    \item Hardhat (Ethereum development framework)
    \item VS Code or another code editor
\end{itemize}

\subsection{Clone the Repository}
\begin{lstlisting}
git clone https://github.com/obios-serra/defi_pokemon
\end{lstlisting}

\subsection{Install Dependencies}
The command \texttt{npm install} installs all the dependencies listed in the project's \texttt{package.json} file, under the "dependencies" and "devDependencies" sections.

\textbf{Backend:}
\begin{lstlisting}
cd defi_pokemon
npm install
\end{lstlisting}

\textbf{Frontend:}
\begin{lstlisting}
cd frontend
npm install
\end{lstlisting}

\subsection{Compile and Deploy Smart Contracts}
\textbf{Start the Hardhat local network:}
\begin{lstlisting}
cd ..
npx hardhat node
\end{lstlisting}

Once the local blockchain is running, a set of pre-funded test accounts will be available. By default, the backend is configured to use \texttt{Account \#0}, which has the following address:

\begin{itemize}
  \item \textbf{Address}: \texttt{0xf39Fd6e51aad88F6F4ce6aB8827279cffFb92266}
  \item \textbf{Balance}: \texttt{10000 ETH}
  \item \textbf{Private key}: \texttt{0xac0974bec39a17e36ba4a6b4d238ff944bacb478cbed5efcae784d7bf4f2ff80}
\end{itemize}

This private key can be imported into MetaMask to access the same account that will deploy the contracts. Doing so allows you to act as the admin within the frontend interface. To import the account in MetaMask, click on the current account on the top of the interface, go to \texttt{+Add account or hardware wallet}, select \texttt{Private Key} as the method, and paste the private key provided above.

\textbf{Deploy the contract:}
\begin{lstlisting}
npx hardhat run scripts/deploy.js --network localhost
\end{lstlisting}

The following message will appear on the Terminal:
\begin{lstlisting}
Compiled 18 Solidity files successfully (evm target: paris).
Deploying contracts with account: 0xf39Fd6e51aad88F6F4ce6aB8827279cffFb92266
PokemonNFT deployed to: 0x5FbDB2315678afecb367f032d93F642f64180aa3
PokemonMarket deployed to: 0xe7f1725E7734CE288F8367e1Bb143E90bb3F0512
\end{lstlisting}


\begin{itemize}
    \item \texttt{0xf39Fd6e51aad88F6F4ce6aB8827279cffFb92266} is the account that deployed the two contracts \texttt{PokemonNFT.sol} and \texttt{PokemonMarket.sol}. This is a local account generated by Hardhat (often account \#0 in the local node). It pays for the gas to deploy the contracts.
	\item \texttt{0x5FbDB2315678afecb367f032d93F642f64180aa3} is the address of the deployed NFT smart contract (for minting and trading Pokémon as NFTs).
	\item \texttt{0xe7f1725E7734CE288F8367e1Bb143E90bb3F0512} is the address of the marketplace contract.
\end{itemize}

\subsection{Configure the Frontend}
Open \texttt{frontend/src/App.js}. Now, copy and paste the deployed NFT and market contract addresses from the output of the contract deployment:

\begin{lstlisting}
const contractAddress = "0xYourNFTContractAddress";
const marketContractAddress = "0xYourMarketContractAddress";
\end{lstlisting}

Just like in \texttt{App.js}, the contract address must also be defined in \texttt{contract.js}. Locate the following line in \texttt{contract.js}:

\begin{lstlisting}
const CONTRACT_ADDRESS = "0xYourNFTContractAddress"
\end{lstlisting}

and replace it with the address of the deployed smart contract (e.g., the output from the deployment script or console). This ensures that the frontend communicates with the correct contract instance on the local network.

\subsection{Run the Frontend}
\begin{lstlisting}
cd frontend
npm start
\end{lstlisting}

Then open your browser and navigate to \texttt{http://localhost:3000/}.

\section{Features}
\subsection{Mint Pokémon NFTs}
\begin{itemize}
    \item Admins can mint Pokémon directly.
    \item Users can mint using a commit-reveal mechanism to prevent front-running.
\end{itemize}

\subsection{Transfer Pokémon}
\begin{itemize}
    \item Users can transfer Pokémon NFTs to other addresses.
    \item Transfers are disabled when the contract is paused.
\end{itemize}

\subsection{Marketplace}
\begin{itemize}
    \item Fixed-price listings: Users can list and buy Pokémon NFTs at a set price.
    \item Auctions: Users can create and participate in auctions for Pokémon NFTs.
    \item Pull-payment model ensures secure fund withdrawals.
\end{itemize}

\section{Security Features}
\begin{itemize}
    \item \textbf{Reentrancy Protection:} Prevents reentrancy attacks using OpenZeppelin's \texttt{ReentrancyGuard}.
    \item \textbf{Access Control:} Admin-only functions are restricted using \texttt{Ownable}.
    \item \textbf{Pause Mechanism:} Admins can pause the contract during emergencies.
    \item \textbf{Commit-Reveal Scheme:} Prevents front-running during minting.
    \item \textbf{Input Validation:} Ensures valid inputs for all functions.
    \item \textbf{Safe Transfers:} Uses \texttt{safeTransferFrom} to prevent token loss.
    \item \textbf{Event Logging:} Emits events for transparency and debugging.
\end{itemize}

\section{Testing}
\begin{itemize}
    \item Run the test suite:
    \begin{lstlisting}
    npx hardhat test
    \end{lstlisting}
    \item Generate a test coverage report:
    \begin{lstlisting}
    npx hardhat coverage
    \end{lstlisting}
\end{itemize}

\section{Technical Support \& References}
\begin{itemize}
    \item \href{https://hardhat.org/}{Hardhat Documentation}
    \item \href{https://docs.openzeppelin.com/contracts/}{OpenZeppelin Contracts}
    \item \href{https://docs.ethers.io/}{Ethers.js}
    \item \href{https://metamask.io/}{MetaMask}
\end{itemize}

\end{document}
